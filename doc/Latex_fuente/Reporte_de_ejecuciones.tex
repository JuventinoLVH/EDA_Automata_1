\documentclass[
	12pt,
	spanish,
]{./recursos/fphw}

% Paqueteria
\usepackage[spanish]{babel}
\usepackage[utf8]{inputenc}     
\usepackage[T1]{fontenc}        
\usepackage{hhline}
\usepackage{booktabs} 
\usepackage{mathpazo}          
\usepackage{caption}           
\usepackage{graphicx} 
\usepackage{multicol}          
\usepackage{enumerate}          
\usepackage{enumitem}
\usepackage{listingsutf8}
\usepackage{amsmath}
\usepackage{subcaption}
\usepackage[table]{xcolor}
\usepackage{tikz}
\usepackage{arydshln}
\graphicspath{ {./recursos/fotos} }

\newcommand{\textbfit}[1]{\textit{\textbf{#1}}}
\newcommand{\figura}[1]{\textbf{figura \ref{fig:#1}}}

\lstset{inputencoding=utf8/latin1}
\definecolor{backcolour}{rgb}{0.97,0.97,0.97}
\lstdefinestyle{mystyle}{
    backgroundcolor=\color{backcolour},   
    commentstyle=\color{gray},
    keywordstyle=\color{magenta},
    stringstyle=\color{purple},
    basicstyle=\ttfamily\footnotesize,
    breakatwhitespace=false,         
    breaklines=true,                 
    frame=single
}
\lstset{style=mystyle}

\newcolumntype{w}{>{\columncolor{white!20!}}c}
\newcolumntype{g}{>{\columncolor{gray!20!}}c}

\begin{document}

% Portada
\title{Automata practica 1}
\author{Luis Juventino Velasquez Hidalgo} 
\begin{titlepage} 
    \center 
    \newcommand{\HRule}{\rule{\linewidth}{0.5mm}} 

    %------------------------------------------------
    %	HEADINGS
    %------------------------------------------------
    \textsc{\LARGE Centro de Investigación en Computación \\ Instituto Politécnico Nacional }\\[1.5cm]
    \textsc{\Large Maestría en Ciencias en Ingeniería de Cómputo}\\[0.5cm]
    \textsc{\large Archivo de especificaciones}\\[0.5cm]

    %------------------------------------------------
    %	Title
    %------------------------------------------------
    \HRule\\[0.4cm]
    \huge\bfseries Automata practica 1\\[0.4cm]
    \HRule\\[1.5cm]

    %------------------------------------------------
    %	Author(s)
    %------------------------------------------------
    %\begin{minipage}{0.4\textwidth}
    %    \begin{flushleft}
    %        \large
    %        \textit{Autor}\\
    %        \textsc{Luis Juventino Velasquez Hidalgo} 
    %    \end{flushleft}
    %\end{minipage}
    %\begin{minipage}{0.4\textwidth}
    %    \begin{flushright}
    %        \large
    %        \textit{Profesor}\\
    %        \textsc{Osvaldo Espinosa Sosa} 
    %    \end{flushright}
    %\end{minipage}

    %------------------------------------------------
    %	Logo
    %------------------------------------------------
    \vfill\vfill
    \includegraphics[width=0.4\textwidth]{CIC.jpg}
     
    %------------------------------------------------
    %	Date
    %------------------------------------------------
    \vfill\vfill\vfill 
    \large\today 
    \vfill 
\end{titlepage}

% ---------------- Introducción ------------------
\section{Introducción}

    El diseño presente en esté documento corresponde a un autómata genérico y sencillo. Tiene como propósito el ser
    usado para practicar el uso de herramientas de verificación y simulación.\\

    \subsection{Descripción del autómata}
    La señal de salida se comporta como un contador, su valor cambia a alto cuando la señal de entrada recibe dos pulsos en alto independiente de la cantidad de ceros entre ellos. Una vez que la salida se activa, el autómata regresa a su estado inicial.\\ 
    Para activar nuevamente la salida, se deben recibir otros dos pulsos en alto en la entrada.\\

    \subsection{Reset}
    El autómata cuenta con una señal de reset sincrónico, activo en alto. Cuando se activa, el autómata pasa a un estado de reposo en donde la salida se encuentra en bajo. Una vez que el reset se desactiva, el autómata pasa a su estado inicial y comienza a contar los pulsos en la entrada.\\
    


% ----------------------------------------------------------------
\section{Entradas y salidas}
    \begin{center} \begin{tabular}{| c | c | c |}
            Señal           & tamaño  &  dirección \\ \hline \hline
            clk             & 1       &  Entrada   \\ \hline
            rst             & 1       &  Entrada   \\ \hline
            din             & 1       &  Entrada   \\ \hline
            dout            & 1       &  Salida    \\ \hline
        \end{tabular}
        \label{tab:tabla_estados}
    \end{center}


% ----------------------------------------------------------------
\section{Reloj, reset y operaciónes}

\subsection{Reloj}

    El autómata es síncrono. La frecuencia del reloj es libre de definir, pero las simulaciones están hechas con un reloj de 100 Hz.

\subsection{rst}

    Se trata de un reset sincrónico, activo en alto. Cuando se activa, el autómata regresa al estado $idle$.

\subsection{din}
    \subsection*{Tabla de estados}
    \begin{center} \begin{tabular}{r || c:c || c:c }
            Estado          & reset   &  din     & Transición       & dout   \\ \hline \hline
            idle            & 1       &  -       & idle             & 0      \\ \hline
            idle            & 0       &  -       & $s_0$            & 0      \\ \hline
            $s_0$           & -       &  0       & $s_0$            & 0      \\ \hline
            $s_0$           & -       &  1       & $s_1$            & 0      \\ \hline
            $s_1$           & -       &  0       & $s_1$            & 0      \\ \hline
            $s_1$           & -       &  1       & $s_0$            & 1      \\ \hline
        \end{tabular}
        \captionof{table}{Tabla de transiciones.}
        \label{tab:tabla_estados}
    \end{center}

\subsection{Diagrama de estados}
    \begin{center} \begin{tikzpicture}[scale=0.2]
            \tikzstyle{every node}+=[inner sep=0pt]
            \draw [black] (10,-10.5) circle (3);
            \draw (10,-10.5) node {$idle$};
            \draw [black] (26.4,-10.5) circle (3);
            \draw (26.4,-10.5) node {$s_0$};
            \draw [black] (26.4,-22.4) circle (3);
            \draw (26.4,-22.4) node {$s_1$};
            \draw [black] (12.961,-10.023) arc (97.11908:82.88092:42.272);
            \fill [black] (23.44,-10.02) -- (22.71,-9.43) -- (22.58,-10.42);
            \draw (18.2,-9.2) node [above] {$reset=0$};
            \draw [black] (8.677,-7.82) arc (234:-54:2.25);
            \draw (10,-3.25) node [above] {$reset=1$};
            \fill [black] (11.32,-7.82) -- (12.2,-7.47) -- (11.39,-6.88);
            \draw [black] (28.86,-12.168) arc (42.33573:-42.33573:6.358);
            \fill [black] (28.86,-12.17) -- (29.03,-13.1) -- (29.77,-12.42);
            \draw (31.02,-16.45) node [right] {$din=1$};
            \draw [black] (23.923,-20.758) arc (-137.12445:-222.87555:6.331);
            \fill [black] (23.92,-20.76) -- (23.75,-19.83) -- (23.01,-20.51);
            \draw (21.73,-16.45) node [left] {$din=1$};
            \draw [black] (25.077,-7.82) arc (234:-54:2.25);
            \draw (26.4,-3.25) node [above] {$din=0$};
            \fill [black] (27.72,-7.82) -- (28.6,-7.47) -- (27.79,-6.88);
            \draw [black] (27.723,-25.08) arc (54:-234:2.25);
            \draw (26.4,-29.65) node [below] {$din=0$};
            \fill [black] (25.08,-25.08) -- (24.2,-25.43) -- (25.01,-26.02);
        \end{tikzpicture}
    \end{center}

% ----------------------------------------------------------------
\end{document}

% Ejemplo de etiquetar una imagen
%        \begin{center}
%            \includegraphics[width=0.5\textwidth]{FFT_Tabla_de_verdad.png}
%            \captionof{figure}{Tabla de verdad del flip-flop tipo T. Recuperado de \cite{libro} }
%            \label{fig:Tabla de verdad FFT}
%        \end{center}

% Ejemplo de cita \cite{libro}
% \textbf{figura \ref{fig:Síntesis del FFT}}

% Ejemplo de como crear una imagen con texto a un lado.
%    \begin{minipage}{0.5\textwidth}
%        Como resultado de la compilación se obtiene el circuito RTL que se muestra a la derecha.
%        La síntesis muestra exactamente lo esperado, tres sumadores exactamente iguales.
%      \end{minipage}
%    \noindent\begin{minipage}{0.4\textwidth}\raggedleft
%        \includegraphics[width=1\|extwidth]{vhdl_pp_rtl_sumadores.png}
%    \end{minipage}%


% Dos imagenes una al lado de otra
%    \begin{multicols}{2} \begin{center} 
%        \includegraphics[width=0.5\textwidth ]{tabla_rc.png}
%        \captionof{figure}{Tabla de verdad de un latch tipo RC.}
%        \label{fig:tt-latch-rc}
%
%    \end{center} \columnbreak \begin{center} 
%        \includegraphics[ width=0.5\textwidth ]{esquema_rc.png}
%        \captionof{figure}{Esquema del latch tipo RC.}
%        \label{fig:latch-rc} 
%   \end{center} \end{multicols}    


% Incrustar código fuente
%   \lstinputlisting[ basewidth=0.5em, caption=circuito.v, language=verilog, numbers=left ]{./recursos/codigo/latch_rc.v}
